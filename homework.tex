\documentclass[12pt]{article}

\usepackage{fouriernc}
\usepackage[T1]{fontenc}
\usepackage{amsmath}
\usepackage{amssymb}
\usepackage[utf8]{inputenc}
\usepackage[english]{babel}
\usepackage{multicol}
\usepackage[margin=0.5in]{geometry}

\setlength{\parindent}{0em}
\setlength{\parskip}{1em}

\newcommand{\curly}[1]{\left\{ #1 \right\}}
\newcommand{\round}[1]{\left( #1 \right)}
\newcommand{\hard}[1]{\left[ #1 \right]}

\title{Cryptography Homework 2}
\author{Tim Harding}

\begin{document}
\maketitle

\section*{1.34}
\subsection*{e}
\subsection*{f}

\section*{1.43}
Given
\begin{align*}
    e_k(m) &\equiv k_1 m + k_2 \pmod{p} \\
    d_k(c) &\equiv k_1^{-1} (c - k_2)
\end{align*}
\subsection*{a}
\textit{Problem:} Given $p = 541$, $k_1 = 34$, and $k_2 = 71$, encrypt $m = 204$ and decrypt $c = 431$.
\textit{Solution:}
\begin{align*}
    e(204) &= 34 \times 204 + 71 \pmod{541} \\
    &\equiv \boxed{515} \\
    \\
    34^{-1} \pmod{541} &\equiv 366 \\
    d(431) &= 336 \times (431 - 71) \pmod{541} \\
    &\equiv \boxed{317}
\end{align*}
\subsection*{c}

\end{document}